\documentclass[12pt,a4paper, openany]{book}
\makeatletter
\setlength{\unitlength}{1cm}
\usepackage[ngerman]{babel}

\usepackage[latin1]{inputenc}
\usepackage{epsfig}
\usepackage{bezier}
\usepackage{xr}
\usepackage{srcltx}
\usepackage{float}
\usepackage{rotating}
\usepackage{array}


\usepackage{amsmath}
\usepackage{amsthm}
\usepackage{amssymb}
\usepackage{amsfonts}

\usepackage{booktabs}  % for table in Latex
\usepackage{theorem}
\usepackage{longtable}
\usepackage{ssm}
\usepackage{pifont}
\usepackage{verbatim}
\usepackage{latexsym}
\usepackage{ifthen}%\externaldocument[LSG-]{loesung}
\usepackage{enumerate}
\usepackage{enumitem} % Customize enumerate, itemize and description environments
\setlist[enumerate]{topsep=2pt, itemsep=5pt}
\setlist[itemize]{topsep=0pt, itemsep=2pt, midpenalty=10000}

\usepackage{psfrag}   % for the letters in .eps figures 

\usepackage{graphics} % for pdf, bitmapped graphics files
\usepackage{graphicx}
\usepackage{lscape}

\usepackage{color}
\definecolor{hell_blau}{RGB}{0, 102, 255}

\usepackage{framed}

\usepackage[e]{esvect} % Darstellung von Vektoren

\usepackage{tabularx}	% Tabellen mit Größenangaben
\newcolumntype{C}[1]{>{\centering\arraybackslash}p{#1}}
\newcolumntype{L}[1]{>{\arraybackslash}p{#1}}

\usepackage{multirow}	% Zellen einer Tabelle in mehrere Zeilen unterteilen

\usepackage{rotating} % Bilder quer auf Seite darstellen
\usepackage{lscape}		% 
\usepackage{fancyhdr}
\setlength{\headheight}{30pt}

\usepackage{setspace}

\usepackage{cancel} % Zum Streichen von Termen in Gleichungen
\renewcommand{\CancelColor}{\color{red}} %change cancel color to red

\usepackage{caption}
\captionsetup{
  justification=centering,
  singlelinecheck=off
}
\usepackage[labelformat=simple]{subcaption}
\renewcommand\thesubfigure{(\alph{subfigure})}


% Zur Formatierung von Titeln usw.:
\usepackage{titlesec}
\titleformat{\chapter}[frame]{\normalfont\LARGE\bfseries}
    {}{5pt}{\centering \sc}
		\titlespacing{\chapter}{0cm}{-2cm}{0cm}

%\usepackage[babel,german=quotes]{csquotes}
\usepackage[
bibstyle=numeric-comp,    % Zitierstil
isbn=false,                % ISBN nicht anzeigen, gleiches geht mit nahezu allen anderen Feldern
%pagetracker=true,          % ebd. bei wiederholten Angaben (false=ausgeschaltet, page=Seite, spread=Doppelseite, true=automatisch)
maxbibnames=50,            % maximale Namen, die im Literaturverzeichnis angezeigt werden (ich wollte alle)
maxcitenames=3,            % maximale Namen, die im Text angezeigt werden, ab 4 wird u.a. nach den ersten Autor angezeigt
autocite=inline,           % regelt Aussehen für \autocite (inline=\parancite)
block=space,               % kleiner horizontaler Platz zwischen den Feldern
backref=true,              % Seiten anzeigen, auf denen die Referenz vorkommt
backrefstyle=three+,       % fasst Seiten zusammen, z.B. S. 2f, 6ff, 7-10
date=short,                % Datumsformat
giveninits=true,
backend=biber]{biblatex}
\addbibresource{Literatur.bib}
\setlength{\bibitemsep}{10pt}
%Autorenformat ändern
%\DeclareNameFormat{default}{%Vollzitate
%  \ifnum\value{listcount}=1\relax
%    \iffirstinits
%      {\usebibmacro{name:last-first}{#1}{#4}{#5}{#7}}
%      {\usebibmacro{name:last-first}{#1}{#3}{#5}{#7}}%
%  \else
%    \iffirstinits
%      {\usebibmacro{name:first-last}{#1}{#4}{#5}{#7}}
%      {\usebibmacro{name:first-last}{#1}{#3}{#5}{#7}}%
%  \fi
%  \usebibmacro{name:andothers}} 
	

\usepackage[babel,german=quotes]{csquotes}
\AtBeginBibliography{%
  \renewcommand*{\mkbibnamelast}[1]{\textsc{#1}}}
	
\usepackage[colorlinks=true, linkcolor=blue, urlcolor=hell_blau, citecolor=red]{hyperref}


%\renewcommand{\textfraction}{0.0}
%\renewcommand{\topfraction}{1.0}
%\renewcommand{\bottomfraction}{1.0}
%\renewcommand{\thefootnote}{\arabic{footnote})~}
%\renewcommand{\labelenumi}{(\arabic{enumi})}
%%%%%%%%%%%%%%%%%%%%%%%%%%%%%%%%%%%%%%%%%%%%%%%%%%%%%%%%%%%%%%%%%%%%%%%%%%%%%
% 	Section for Page Layout
%%%%%%%%%%%%%%%%%%%%%%%%%%%%%%%%%%%%%%%%%%%%%%%%%%%%%%%%%%%%%%%%%%%%%%%%%%%%%
\singlespacing
%\onehalfspacing
%\doublespacing

\pagestyle{plain}

%%%%%%%%%%%%%%%%%%%%%
% 	all about the text and page style
%		see Latex_layout.svg
%   1. one inch + \hoffset
%   2. one inch + \voffset
%   3. \oddsidemargin = 31pt
%   4. \topmargin = 20pt
%   5. \headheight = 12pt
%   6. \headsep = 25pt
%   7. \textheight = 592pt
%   8. \textwidth = 390pt
%   9. \marginparsep = 10pt
%  10. \marginparwidth = 35pt
%  11. \footskip = 30pt

%		DIN A4 : height x width: 29.7 cm x 21.0 cm
%		1 inch = 2.54 cm
%		define page layout using package "`setlength"'
\setlength{\hoffset}{-0.54cm}  
\setlength{\voffset}{-0.54cm}  
\setlength{\oddsidemargin}{+0.3cm}
\setlength{\evensidemargin}{-0.3cm} 
\setlength{\topmargin}{0.0cm}
\setlength{\headheight}{0.6cm}
\setlength{\headsep}{1cm}
\setlength{\textheight}{23.6cm}
\setlength{\textwidth}{17.0cm}
\setlength{\marginparsep}{0.0cm}  
\setlength{\marginparwidth}{0cm}  
\setlength{\footskip}{1cm}

% additional outlay definitions
%	no indent for new paragraph by \\
\setlength{\parindent}{0mm}
\setlength{\marginparpush}{0cm}  

%%%%%%%%%%%%%%%%%%%%%%%%%%%%%%%%%%%%%%%%%%%%%%%%%%%%%%%%%%%%%%



%%%%%%%%%%%%%%%%%%%%%%%%%%%%%%%%%%%%%%%%%%%%%%%%%%%%%%%%%%%%%%%%%%%%%%%%%%%%%
%    	Set line spacing  (by Multiplier ... 1.5 here)
%%%%%%%%%%%%%%%%%%%%%%%%%%%%%%%%%%%%%%%%%%%%%%%%%%%%%%%%%%%%%%%%%%%%%%%%%%%%%
%\renewcommand{\baselinestretch}{1.05}
%%%%%%%%%%%%%%%%%%%%%%%%%%%%%%%%%%%%%%%%%%%%%%%%%%%%%%%%%%%%%%%%%%%%%%%%%%%%%

%%%%%%%%%%%%%%%%%%%%%%%%%%%%%%%%%%%%%%%%%%%%%%%%%%%%%%%%%%%%%%%%%%%%%%%%%%%%%
%    	Fancy Page Layout with Fancy Header
%%%%%%%%%%%%%%%%%%%%%%%%%%%%%%%%%%%%%%%%%%%%%%%%%%%%%%%%%%%%%%%%%%%%%%%%%%%%%
\pagestyle{fancy}

\let\origdoublepage\cleardoublepage
\newcommand{\clearemptydoublepage}{%
  \clearpage
  {\pagestyle{empty}\origdoublepage}%
}
%%%%%%%%%%%%%%%%%%%%%%%%%%%%%%%%%%%%%%%%%%%%%%%%%%%%%%%%%%%%%%%%%%%%%%%%%%%%%

%%%%%%%%%%%%%%%%%%%%%%%%%%%%%%%%%%%%%%%%%%%%%%%%%%%%%%%%%%%%%%%%%%%%%%%%%%%%%
% 	all about head & foot rule 
\renewcommand{\headrulewidth}{0.5pt}
\renewcommand{\footrulewidth}{0.5pt}
\setlength{\headwidth}{\textwidth}

%\renewcommand{\rightmark}{\thesection\ #1}
%\renewcommand{\leftmark}{Chapter \thechapter}
%\renewcommand{\chaptermark}[1]{\markboth{Chapter~\thechapter:\ #1}{}}
\renewcommand{\sectionmark}[1]{\markboth{\textsc{\thesection.~#1}}{}}
\renewcommand{\subsectionmark}[1]{\markright{\textsc{\thesubsection.~#1}}}
%\renewcommand{\sectionmark}[1]{\markright{\thesection.~#1}{}}
%%%%%%%%%%%%%%%%%%%%%%%%%%%%%%%%%%%%%%%%%%%%%%%%%%%%%%%%%%%%%%%%%%%%%%%%%%%%%

%%%%%%%%%%%%%%%%%%%%%%%%%%%%%%%%%%%%%%%%%%%%%%%%%%%%%%%%%%%%%%%%%%%%%%%%%%%%%
% 	all about shown content in header
%\lhead[\bfseries Page \thepage]{\bfseries\rightmark}
\lhead[\fancyplain{}{\leftmark}]{\fancyplain{}{}}
%\rhead[\bfseries\leftmark]{\bfseries Page \thepage}
\rhead[\fancyplain{}{}]{\fancyplain{}{\rightmark}}
\cfoot{--~\thepage~--}
% \rfoot{\bfseries\thepage}
%%%%%%%%%%%%%%%%%%%%%%%%%%%%%%%%%%%%%%%%%%%%%%%%%%%%%%%%%%%%%%%%%%%%%%%%%%%%%

\fancypagestyle{plain}{%
\fancyhf{} % clear all header and footer fields
\cfoot{--~\thepage~--} % except the center
\renewcommand{\headrulewidth}{0pt}
\renewcommand{\footrulewidth}{0.5pt}}


%%%%%%%%%%%%%%%%%%%%%%%%%%%%%%% AMSTHM new commands %%%%%%%%%%%%%%%%%%%%%%%%%%
%\newtheoremstyle{MyPlain}% name
%  {9pt}%      Space above, empty = `usual value'
% 	{9pt}%      Space below
%  {\itshape}% Body font
%  {\parindent}%         Indent amount (empty = no indent, \parindent = para indent)
%  {\bfseries}% Thm head font
%  {}%        Punctuation after thm head
%  {0.5em}% Space after thm head: \newline = linebreak
%  {}%         Thm head spec 
%
%\theoremstyle{MyPlain} %default[chapter]

%\newtheorem{theorem}{Theorem}[Section]
%\newtheorem{proposition}[theorem]{Proposition}
%\newtheorem{lemma}[theorem]{Lemma}
%\newtheorem{corollary}[theorem]{Corollary}
%\newtheorem{definition}[theorem]{Definition}
%%\newtheorem*{definition*}{Definition}
%\newtheorem{remark}[theorem]{Remark}
%\newtheorem{claim}[theorem]{Claim}
%\newtheorem{example}[theorem]{Example}
%\newtheorem{examples}[theorem]{Examples}
%\newtheorem{assumption}[theorem]{Assumption}
%\newtheorem{motivation}[theorem]{Motivation}
%\newtheorem{property}{Property}
%\newtheorem{note}[theorem]{Note}
 %
%%%%%%%%%%%%%%%%%%%%%%%%%%%%%%% AMSTHM new commands %%%%%%%%%%%%%%%%%%%%%%%%%%
%
%
%%%%%%%%%%%%%%%%%%%%%%%%%%%%%%%%%%%%%%%%%%%%%%%%%%%%%%%%%%%%%%%%%%%%%%%%%%%%%
%	(RE-)DEFINE NEW MATH EXPRESSIONS

%	Define new math operators
%\DeclareMathOperator{\kern}{kern} %somewhere already defined
\DeclareMathOperator{\essinf}{ess-inf}
\DeclareMathOperator*{\esssup}{ess-sup}
\DeclareMathOperator*{\supr}{sup}
\DeclareMathOperator{\dist}{dist}
\DeclareMathOperator{\diag}{diag}
\DeclareMathOperator{\graph}{graph}
\DeclareMathOperator{\sign}{sign}
\DeclareMathOperator{\rank}{rank}
\DeclareMathOperator{\sat}{sat}
\DeclareMathOperator{\spec}{spec}
\DeclareMathOperator{\loc}{loc}
\DeclareMathOperator{\grad}{grad}
\DeclareMathOperator{\fin}{fin}
\DeclareMathOperator{\vol}{vol}
\DeclareMathOperator{\col}{col}
\DeclareMathOperator{\row}{row}
\DeclareMathOperator{\lin}{lin}

%	Abbreviate sets, compact sets  (of numbers, ...)
\newcommand{\N}{\mathbb{N}}
\newcommand{\B}{\mathbb{B}}
\newcommand{\Z}{\mathbb{Z}}
\newcommand{\Q}{\mathbb{Q}}
\newcommand{\R}{\mathbb{R}}
\newcommand{\Rp}{\mathbb{R}_{\geq 0}}
\newcommand{\Rsp}{\mathbb{R}_{> 0}}
\newcommand{\C}{\mathbb{C}}
\newcommand{\Cn}{\mathbb{C}_{< 0}}
\newcommand{\Czp}{\mathbb{C}_{\geq 0}}
%	compact set
\newcommand{\cset}[1]{\mathfrak{#1}}
%	open set
\newcommand{\oset}[1]{\mathcal{#1}}
%	function set
\newcommand{\fset}[1]{\mathcal{#1}}
%	system class 
\newcommand{\sclass}[1]{\mathcal{#1}}
%	operator class 
\newcommand{\oclass}[1]{\mathcal{#1}}
%	boundary set
\newcommand{\bset}[1]{\mathcal{#1}}

%	often used symbols or nomenclature

%	special for Praktikum ET S2
\newcommand{\eff}{\rm{eff}}
\newcommand{\str}{\rm{STR}}


\newcommand{\myref}{\rm{ref}}
%	my (gain) scaling
\newcommand{\gsc}{\varsigma}
\newcommand{\gscmin}{\underline{\varsigma}}
%	high-frequency gain
\newcommand{\hfg}{\gamma_0}
%	steady-state gain
\newcommand{\ssg}{\gamma_{\infty}}
%	root locus center
\newcommand{\rlc}{\Xi}
%	boundaries
%\newcommand{\pF}{\mathcal{F}}
\newcommand{\pF}{\psi}

\newcommand{\pFE}{\psi_E}
\newcommand{\pFphi}{\mathcal{F}_{\phi}}
\newcommand{\dx}[1]{{\rm \, d} #1 \,}
\newcommand{\fdiff}[2]{\frac{\textrm{d}^{#1}}{\textrm{d}#2^{#1}}\,}
\newcommand{\fpartial}[2]{\frac{\partial^{#1}}{\partial #2^{#1}}\,}
\newcommand{\fpartials}[3]{\frac{\partial^{#1}{#3}}{\partial #2^{#1}}\,}
\newcommand{\eps}{\varepsilon}
\newcommand{\foraa}{\textrm{for~a.a.~}}
%	certain instants of time
\newcommand{\tstar}{t^{\star}}
\newcommand{\tlstar}{t_{\star}}
%	motion control objectives
\newcommand{\trise}[1]{t_{y(\cdot),{#1}}^{r}}
\newcommand{\triseref}[1]{t_{{\myref},{#1}}^{r}}
\newcommand{\tset}[1]{t_{y(\cdot),{#1}}^{s}}
\newcommand{\tsetref}[1]{t_{{\myref},{#1}}^{s}}
\newcommand{\os}{\Delta_{y(\cdot)}^{os}}
\newcommand{\osref}{\Delta_{\myref}^{os}}
%	gains 
\newcommand{\kmax}{k_{\max}}
\newcommand{\kinf}{k_{\infty}}
\newcommand{\kmin}{k_{\min}}
\newcommand{\kstar}{k^{\star}}
% control
\newcommand{\ufeas}{u_{\rm{feas}}}
\newcommand{\uhat}{\hat{u}}

\newcommand{\setdef}[2]{\left\{\ #1\ \left|\vphantom{#1}\ #2\ \right.\right\}}

%	function spaces
\newcommand{\Lp}[1]{\fset{L}^{#1}}
\newcommand{\Lone}{\fset{L}^{1}}
\newcommand{\Loneloc}{\fset{L}^{1}_{\loc}}
\newcommand{\Ltwo}{\fset{L}^{2}}
\newcommand{\Linf}{\fset{L}^{\infty}}
\newcommand{\Linfloc}{\fset{L}^{\infty}_{\loc}}
\newcommand{\Ck}{\fset{C}^{k}}
\newcommand{\Czero}{\fset{C}}
\newcommand{\Cone}{\fset{C}^{1}}
\newcommand{\Cinf}{\fset{C}^{\infty}}
\newcommand{\Wkinf}{\fset{W}^{k,\infty}}
\newcommand{\Woneinf}{\fset{W}^{1,\infty}}
\newcommand{\Wtwoinf}{\fset{W}^{2,\infty}}


%	lapace
\newcommand{\laplaceT}[1]{\mathscr{L}\left\{#1\right\}}

%	myVector = \mv
\newcommand{\mv}[1]{\boldsymbol{#1}}
%	myMatrix = \mm
\newcommand{\mm}[1]{\boldsymbol{#1}}
%	myOperator = \mO
\newcommand{\mo}[1]{\boldsymbol{\mathfrak{#1}}}
%	norm ||.||
\newcommand{\norm}[1]{\left\lVert #1 \right\rVert}
% ||.||_infty
\newcommand{\esnorm}[1]{\lVert #1 \rVert_{\infty}}
\newcommand{\Lnorm}[2]{\lVert #2 \rVert_{\fset{L}^{#1}}}
%	inner product <.,.>
\newcommand{\iproduct}[2]{\langle #1,\, #2 \rangle}
%	for dimensions e.g. [Nm]
\newcommand{\di}[1]{\left[\rm{#1}\right]}

% for typewriting:
\newcommand{\tw}[1]{{\tt #1}}

%	system classes
\newcommand{\Sonelin}{\sclass{S}_1^{\lin}}
\newcommand{\Stwolin}{\sclass{S}_2^{\lin}}
\newcommand{\Sone}{\sclass{S}_1}
\newcommand{\Stwo}{\sclass{S}_2}

%	Dimensions
\newcommand{\Ohm}{\Omega}
%
%	CHANGE ENUMERATE NUMBERING
\renewcommand{\labelenumi}{\arabic{enumi}.)}
%\renewcommand{\labelenumii}{(\alph{enumii})}
% 	

% define colors for TIKZ
\definecolor{brown}{RGB}{153, 51,0} %	from Matlab (0.6*255, 0.2*255,0)
\definecolor{orange}{RGB}{255, 127.5,0}
\definecolor{gray}{RGB}{127.5, 127.5,127.5}
%\definecolor{r2}{RGB}{252,177,49}
%\definecolor{r3}{RGB}{35,34,35}
%\definecolor{r4}{RGB}{0,157,87}
%\definecolor{r5}{RGB}{238,50,78}


%	define macros to draw colored lines with TIKZ package
\newcommand{\blueline}{\protect\tikz{\protect\draw[very thick,blue] (0,-0.5ex)(0,0)--(4ex,0);}}
\newcommand{\orangeline}{\protect\tikz{\protect\draw[very thick,orange] (0,-0.5ex)(0,0)--(4ex,0);}}
\newcommand{\brownline}{\protect\tikz{\protect\draw[very thick,brown] (0,-0.5ex)(0,0)--(4ex,0);}}
\newcommand{\bluedashdottedline}{\protect\tikz{\protect\draw[very thick,blue,dashdotted] (0,-0.5ex)(0,0)--(4ex,0);}}
\newcommand{\bluedottedline}{\protect\tikz{\protect\draw[very thick,blue,dotted] (0,-0.5ex)(0,0)--(4ex,0);}}
\newcommand{\bluedashedline}{\protect\tikz{\protect\draw[very thick,dashed,blue] (0,-0.5ex)(0,0)--(4ex,0);}}
\newcommand{\cyanline}{\protect\tikz{\protect\draw[very thick,cyan] (0,-0.5ex)(0,0)--(4ex,0);}}
\newcommand{\cyandashdottedline}{\protect\tikz{\protect\draw[very thick,dashdotted,cyan] (0,-0.5ex)(0,0)--(4ex,0);}}
\newcommand{\cyandashedline}{\protect\tikz{\protect\draw[very thick,dashed,cyan] (0,-0.5ex)(0,0)--(4ex,0);}}
\newcommand{\cyandottedline}{\protect\tikz{\protect\draw[very thick,dotted,cyan] (0,-0.5ex)(0,0)--(4ex,0);}}
\newcommand{\magentaline}{\protect\tikz{\protect\draw[very thick,magenta] (0,-0.5ex)(0,0)--(4ex,0);}}
\newcommand{\magentadashedline}{\protect\tikz{\protect\draw[very thick, dashed,magenta] (0,-0.5ex)(0,0)--(4ex,0);}}
\newcommand{\magentdaottedline}{\protect\tikz{\protect\draw[very thick, dotted,magenta] (0,-0.5ex)(0,0)--(4ex,0);}}
\newcommand{\magentadashdottedline}{\protect\tikz{\protect\draw[very thick, dashdotted,magenta] (0,-0.5ex)(0,0)--(4ex,0);}}
\newcommand{\greenline}{\protect\tikz{\protect\draw[very thick,green] (0,-0.5ex)(0,0)--(4ex,0);}}
\newcommand{\blackline}{\protect\tikz{\protect\draw[very thick,black] (0,-0.5ex)(0,0)--(4ex,0);}}
\newcommand{\blackdashedline}{\protect\tikz{\protect\draw[very thick,dashed,black] (0,-0.5ex)(0,0)--(4ex,0);}}
\newcommand{\blackdottedline}{\protect\tikz{\protect\draw[very thick,dotted,black] (0,-0.5ex)(0,0)--(4ex,0);}}
\newcommand{\blackdashdottedline}{\protect\tikz{\protect\draw[very thick,black,dashdotted] (0,-0.5ex)(0,0)--(4ex,0);}}
\newcommand{\redline}{\protect\tikz{\protect\draw[very thick,red] (0,-0.5ex)(0,0)--(4ex,0);}}
\newcommand{\reddottedline}{\protect\tikz{\protect\draw[very thick,dotted,red] (0,-0.5ex)(0,0)--(4ex,0);}}
\newcommand{\reddashedline}{\protect\tikz{\protect\draw[very thick,dashed,red] (0,-0.5ex)(0,0)--(4ex,0);}}

\newcommand{\graydashedline}{\protect\tikz{\protect\draw[very thick,dashed,gray] (0,-0.5ex)(0,0)--(4ex,0);}}
\newcommand{\graydashdottedline}{\protect\tikz{\protect\draw[very thick,dashdotted,gray] (0,-0.5ex)(0,0)--(4ex,0);}}

%	to include colored comments
\def\ch#1{\textcolor[rgb]{1.00,0.00,0.00}{#1}}

\newcommand{\vect}[1]{\vec{#1}}
\newcommand{\mtx}[1]{{\bf #1}}
\newcommand{\rz}[1]{\underline{#1}} % Raumzeiger
\newcommand{\Clarke}[1]{\mathcal{T_C}\left(#1\right)} % Clarke-Transformation
\newcounter{findex}


\listfiles
\begin{document}
\pagenumbering{roman}

\begin{titlepage}

\vspace*{-2cm}
\begin{minipage}[t]{17.0cm}
	\begin{center}%
		
		\begin{picture}(17.0,3.0)(0,0)
			
			\linethickness{0.04cm}
			
			\put(0.0 ,1.54){\line(1,0){13.3}}
			
			\put(17.2,1.50){\makebox(0,0)[cr]{\includegraphics[height=2.0cm]{./figures/TUM_Logo.eps}}}
			
			\put(0.1,2.73){\makebox(13,0)[tl]{\textbf{Chair of High-Power Converter Systems}}}
			
			\put(0.1,1.71){\makebox(13,0)[bl]{Technical University of Munich \hfill Prof. Marcelo Lobo Heldwein}}
			
			\put(0.1,1.29){\makebox(13,0)[tl]{\renewcommand{\arraystretch}{1.02}
					{\small
						\begin{tabular}[t]{@{}l@{}}%
							Arcisstr. 21            \\
							80333 Munich
					\end{tabular}}
					\hfill
					{\small 
						\begin{tabular}[t]{@{}l@{\,\,\,}l@{}}
							Email: & hlu@ed.tum.de     \\
							Web: & www.epe.ed.tum.de/hlu
					\end{tabular}}
					\hfill
					{\small	
						\begin{tabular}[t]{@{}l@{\,\,\,}l@{}}
							Tel.: & +49 (0)89 289 28358   \\
							Fax:  & +49 (0)89 289 28336
					\end{tabular}}
			}}
			\thinlines
		\end{picture}
	\end{center}
\end{minipage}


\vspace*{3cm}

\begin{center}
{\LARGE \sc Practical Course \\[4ex]}
{\LARGE \bf Simulation and Optimization of  \\[5pt]
Mechatronic Drive Systems for MSPE} \\[8ex]
Summer Semester 2024 
\end{center}

\vfill

\begin{flushleft}

	Haoheng Li / 03797669 \\
\end{flushleft}

\end{titlepage}

\clearpage


%\tableofcontents

\pagenumbering{arabic}
\setcounter{chapter}{0} % command for changing the chapter counter
\chapter{Experiment 1 \\ Model of the DC Machine}

 \setcounter{section}{4} % command for changing the section counter
\section{Model of the DC Machine}

\begin{enumerate}
\item {\bf *}  The Laplace transform of euquations (1.5) are:
\begin{align}
    U_A(s) &= E_A(s) + R_A I_A(s) + s L_A I_A(s) \tag{1.5a} \\[6pt]
    E_A(s) &= C_M \Psi_E(s)\,\Omega_M(s) \tag{1.5b} \\[6pt]
    M_M(s) &= C_M \Psi_E(s)\, I_A(s) \tag{1.5c} \\[6pt]
    U_E(s) &= R_E I_E(s) + s\, \Psi_E(s) \tag{1.5d} \\[6pt]
    \Psi_E(s) &= f(I_E(s)) \tag{1.5e} \\[6pt]
    M_M(s) - M_L(s) &= s \Theta_M  \, \Omega_M(s) \tag{1.5f}
\end{align}

The signal flow graph is shown in Figure \ref{fig:Signal Flow Graph of the DC Machine}.
\begin{figure}[H]
    \centering
    \includegraphics[width=0.8\textwidth]{figures/block diagram of DC machine.png}
    \caption{Signal Flow Graph of the DC Machine}
    \label{fig:Signal Flow Graph of the DC Machine}
\end{figure}    
\item {\bf *} 
The four wortking moed of the DC machine are shown in Figure \ref{fig:Four_working_modes_of_the_DC_machine}.
\begin{figure}[H]
    \centering
    \includegraphics[width=0.8\textwidth]{figures/Four working modes of the DC machine.png}
    \caption{Four working modes of the DC machine}
    \label{fig:Four_working_modes_of_the_DC_machine}
\end{figure}

In a steady state, the load torque of working mode II and III must be positive. 
\item {\bf *} 
The operating point of maximum mechanical power output is shown in Figure \ref{fig:Operating point of maximum mechanical power output}.
\begin{figure}[H]
    \centering
    \includegraphics[width=0.6\textwidth]{figures/Operating point of maximum mechanical power output.png}
    \caption{Operating point of maximum mechanical power output}
    \label{fig:Operating point of maximum mechanical power output}
\end{figure}

The function of $ \Omega_M(M_M)$ can be derived as follows:
\begin{equation}
    \Omega_M(M_M) = \Omega_{M0}\left(1 - \frac{M_M}{M_S}\right)
\end{equation}
    
the mechanical power output $P_M$ is:
\begin{equation}
    P_M(M_M) = \Omega_M(M_M) M_M = \Omega_{M0}\left(1 - \frac{M_M}{M_S}\right) M_M
\end{equation}

To determine the maximum power, we differentiate with respect to the torque:
\begin{equation}
    \frac{dP_M}{dM_M} = \Omega_{M0}\left(1 - \frac{2M_M}{M_S}\right) = 0
\end{equation}

when $M_M = \frac{1}{2} M_S$, the mechanical power output is maximum, which is $\frac{M_S \, \Omega_{M0}}{4}$.
\end{enumerate}


\subsection{Behavior of the DC Machine}

\begin{enumerate}
\setcounter{enumi}{3} % command for changing the enumeration counter
\item {\bf *} 

\begin{enumerate}
\item {\bf} 
The nominal torque $M_{MN}$ can be calculated as follows:
\begin{equation}
    M_{MN} = \frac{P_{N}}{\Omega_{MN}} = \frac{200W}{209.44rad\/s} = 0.955 N\cdot m
\end{equation}

The excitation resistance $R_E$ can be calculated as follows:
\begin{equation}
    R_E = \frac{U_{EN}}{I_{EN}} = \frac{220V}{0.1A} = 2.2k\Omega
\end{equation}
\item {\bf} 
The armature resistance $R_A$ and armature inductance $L_A$ can be calculated from the step response of the armature current $I_A(t)$, where $I_A(t)$ can be expressed as follows:
\begin{equation}
    I_A(t) = \frac{U_A}{R_A} \left(1 - e^{-\frac{t}{\tau}}\right)
\end{equation}
, where $\tau = \frac{L_A}{R_A}$ is the time constant.

The armature resistance $R_A$ can be calculated as follows:
\begin{equation}
    R_A = \frac{U_A}{I_{A\infty}} = \frac{U_A}{I_A}= 22 \Omega
\end{equation}

The armature inductance $L_A$ can be calculated as follows:
\begin{equation}
    L_A = \tau R_A = 0.017s \times 22 \Omega = 374mH    
\end{equation}
\item {\bf}  the machine constant $C_M$ can be calculated as follows:
\begin{equation}
    C_M = \frac{U_{AN} - R_A I_{AN}}{\Psi_E \Omega_{MN}} = \frac{220V-1A\times 22\Omega}{1Vs \times 209.44rad/s} = 0.96
\end{equation}
\item {\bf} 
The derivative of the angular velocity during acceleration and deceleration can be calculated as follows:
\begin{equation}
    \begin{cases}
        (\frac{d\Omega_M}{dt})_a = \frac{M_M - M_L}{\Theta_M} \\
        (\frac{d\Omega_M}{dt})_d = \frac{-M_M-M_L}{\Theta_M} \\
        M_M = C_M \Psi_E I_A
    \end{cases}
\end{equation}

The moment of inertia $\Theta_M$ can be calculated as follows:
\begin{equation}
    \Theta_M = \frac{2M_M}{(\frac{d\Omega_M}{dt})_a - (\frac{d\Omega_M}{dt})_d} = 
    \frac{2\times 0.96 \times 0.5A \times 1Vs}{756.31rad/s^2} = 1.3 g\cdot m^2
\end{equation}
\end{enumerate}
\end{enumerate}
\subsection{Converter Supplied Operation}

\begin{enumerate}
\setcounter{enumi}{9} % command for changing the enumeration counter
\item {\bf *} The maximum carrier frequency can be calculated as follows:
\begin{equation}
    f_{c,max} = 0.5r_{max} = 1kHz
\end{equation}
\item {\bf *} The waveform of the command signals when $U^*_A=0V$ is shown in Figure \ref{fig:Waveform of the command signals when $U^*_A=0V$}.
\begin{figure}[H]
    \centering
    \includegraphics[width=0.5\textwidth]{figures/PWM.png}
    \caption{Waveform of the command signals when $U^*_A=0V$}
    \label{fig:Waveform of the command signals when $U^*_A=0V$}
\end{figure}
\end{enumerate}

\begin{enumerate}
\setcounter{enumi}{12} % command for changing the enumeration counter
\item {\bf *} The transfer function of the power elctronic converter can be written as follows:
\begin{equation}
    \frac{U_A(s)}{U^*_A(s)} = \frac{U_{dc}}{1 + sT_{PE}}
\end{equation}
, where $T_{PE}=\frac{1}{f_c}$ is the time constant of the power electronic converter.
\end{enumerate}


\end{document}
