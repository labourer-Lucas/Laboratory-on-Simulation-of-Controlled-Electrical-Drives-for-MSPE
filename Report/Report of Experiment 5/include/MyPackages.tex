\usepackage[english]{babel}

\usepackage[latin1]{inputenc}
\usepackage{epsfig}
\usepackage{bezier}
\usepackage{xr}
\usepackage{srcltx}
\usepackage{float}
\usepackage{rotating}
\usepackage{array}


\usepackage{amsmath}
\usepackage{amsthm}
\usepackage{amssymb}
\usepackage{amsfonts}

\usepackage{booktabs}  % for table in Latex
\usepackage{theorem}
\usepackage{longtable}
\usepackage{ssm}
\usepackage{pifont}
\usepackage{verbatim}
\usepackage{latexsym}
\usepackage{ifthen}%\externaldocument[LSG-]{loesung}
\usepackage{enumerate}
\usepackage{enumitem} % Customize enumerate, itemize and description environments
\setlist[enumerate]{topsep=2pt, itemsep=5pt}
\setlist[itemize]{topsep=0pt, itemsep=2pt, midpenalty=10000}

\usepackage{psfrag}   % for the letters in .eps figures 

\usepackage{graphics} % for pdf, bitmapped graphics files
\usepackage{graphicx}
\usepackage{lscape}
\usepackage{tikz}
\usepackage{color}
\definecolor{hell_blau}{RGB}{0, 102, 255}

\usepackage{framed}

\usepackage[e]{esvect} % Darstellung von Vektoren

\usepackage{tabularx}	% Tabellen mit Größenangaben
\newcolumntype{C}[1]{>{\centering\arraybackslash}p{#1}}
\newcolumntype{L}[1]{>{\arraybackslash}p{#1}}

\usepackage{multirow}	% Zellen einer Tabelle in mehrere Zeilen unterteilen

\usepackage{rotating} % Bilder quer auf Seite darstellen
\usepackage{lscape}		% 
\usepackage{fancyhdr}
\setlength{\headheight}{30pt}

\usepackage{setspace}

\usepackage{cancel} % Zum Streichen von Termen in Gleichungen
\renewcommand{\CancelColor}{\color{red}} %change cancel color to red

\usepackage{caption}
\captionsetup{
  justification=centering,
  singlelinecheck=off
}
\usepackage[labelformat=simple]{subcaption}
\renewcommand\thesubfigure{(\alph{subfigure})}


% Zur Formatierung von Titeln usw.:
\usepackage{titlesec}
\titleformat{\chapter}[frame]{\normalfont\LARGE\bfseries}
    {}{5pt}{\centering \sc}
		\titlespacing{\chapter}{0cm}{-2cm}{0cm}

%\usepackage[babel,german=quotes]{csquotes}
\usepackage[
bibstyle=numeric-comp,    % Zitierstil
isbn=false,                % ISBN nicht anzeigen, gleiches geht mit nahezu allen anderen Feldern
%pagetracker=true,          % ebd. bei wiederholten Angaben (false=ausgeschaltet, page=Seite, spread=Doppelseite, true=automatisch)
maxbibnames=50,            % maximale Namen, die im Literaturverzeichnis angezeigt werden (ich wollte alle)
maxcitenames=3,            % maximale Namen, die im Text angezeigt werden, ab 4 wird u.a. nach den ersten Autor angezeigt
autocite=inline,           % regelt Aussehen für \autocite (inline=\parancite)
block=space,               % kleiner horizontaler Platz zwischen den Feldern
backref=true,              % Seiten anzeigen, auf denen die Referenz vorkommt
backrefstyle=three+,       % fasst Seiten zusammen, z.B. S. 2f, 6ff, 7-10
date=short,                % Datumsformat
giveninits=true,
backend=biber]{biblatex}
\addbibresource{Literatur.bib}
\setlength{\bibitemsep}{10pt}
%Autorenformat ändern
%\DeclareNameFormat{default}{%Vollzitate
%  \ifnum\value{listcount}=1\relax
%    \iffirstinits
%      {\usebibmacro{name:last-first}{#1}{#4}{#5}{#7}}
%      {\usebibmacro{name:last-first}{#1}{#3}{#5}{#7}}%
%  \else
%    \iffirstinits
%      {\usebibmacro{name:first-last}{#1}{#4}{#5}{#7}}
%      {\usebibmacro{name:first-last}{#1}{#3}{#5}{#7}}%
%  \fi
%  \usebibmacro{name:andothers}} 
	

\usepackage[babel,german=quotes]{csquotes}
\AtBeginBibliography{%
  \renewcommand*{\mkbibnamelast}[1]{\textsc{#1}}}
	
\usepackage[colorlinks=true, linkcolor=blue, urlcolor=hell_blau, citecolor=red]{hyperref}


%\renewcommand{\textfraction}{0.0}
%\renewcommand{\topfraction}{1.0}
%\renewcommand{\bottomfraction}{1.0}
%\renewcommand{\thefootnote}{\arabic{footnote})~}
%\renewcommand{\labelenumi}{(\arabic{enumi})}