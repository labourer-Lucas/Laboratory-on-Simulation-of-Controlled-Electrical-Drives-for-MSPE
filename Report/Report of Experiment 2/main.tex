\documentclass[12pt,a4paper, openany]{book}
\makeatletter
\setlength{\unitlength}{1cm}
\usepackage[ngerman]{babel}

\usepackage[latin1]{inputenc}
\usepackage{epsfig}
\usepackage{bezier}
\usepackage{xr}
\usepackage{srcltx}
\usepackage{float}
\usepackage{rotating}
\usepackage{array}


\usepackage{amsmath}
\usepackage{amsthm}
\usepackage{amssymb}
\usepackage{amsfonts}

\usepackage{booktabs}  % for table in Latex
\usepackage{theorem}
\usepackage{longtable}
\usepackage{ssm}
\usepackage{pifont}
\usepackage{verbatim}
\usepackage{latexsym}
\usepackage{ifthen}%\externaldocument[LSG-]{loesung}
\usepackage{enumerate}
\usepackage{enumitem} % Customize enumerate, itemize and description environments
\setlist[enumerate]{topsep=2pt, itemsep=5pt}
\setlist[itemize]{topsep=0pt, itemsep=2pt, midpenalty=10000}

\usepackage{psfrag}   % for the letters in .eps figures 

\usepackage{graphics} % for pdf, bitmapped graphics files
\usepackage{graphicx}
\usepackage{lscape}

\usepackage{color}
\definecolor{hell_blau}{RGB}{0, 102, 255}

\usepackage{framed}

\usepackage[e]{esvect} % Darstellung von Vektoren

\usepackage{tabularx}	% Tabellen mit Größenangaben
\newcolumntype{C}[1]{>{\centering\arraybackslash}p{#1}}
\newcolumntype{L}[1]{>{\arraybackslash}p{#1}}

\usepackage{multirow}	% Zellen einer Tabelle in mehrere Zeilen unterteilen

\usepackage{rotating} % Bilder quer auf Seite darstellen
\usepackage{lscape}		% 
\usepackage{fancyhdr}
\setlength{\headheight}{30pt}

\usepackage{setspace}

\usepackage{cancel} % Zum Streichen von Termen in Gleichungen
\renewcommand{\CancelColor}{\color{red}} %change cancel color to red

\usepackage{caption}
\captionsetup{
  justification=centering,
  singlelinecheck=off
}
\usepackage[labelformat=simple]{subcaption}
\renewcommand\thesubfigure{(\alph{subfigure})}


% Zur Formatierung von Titeln usw.:
\usepackage{titlesec}
\titleformat{\chapter}[frame]{\normalfont\LARGE\bfseries}
    {}{5pt}{\centering \sc}
		\titlespacing{\chapter}{0cm}{-2cm}{0cm}

%\usepackage[babel,german=quotes]{csquotes}
\usepackage[
bibstyle=numeric-comp,    % Zitierstil
isbn=false,                % ISBN nicht anzeigen, gleiches geht mit nahezu allen anderen Feldern
%pagetracker=true,          % ebd. bei wiederholten Angaben (false=ausgeschaltet, page=Seite, spread=Doppelseite, true=automatisch)
maxbibnames=50,            % maximale Namen, die im Literaturverzeichnis angezeigt werden (ich wollte alle)
maxcitenames=3,            % maximale Namen, die im Text angezeigt werden, ab 4 wird u.a. nach den ersten Autor angezeigt
autocite=inline,           % regelt Aussehen für \autocite (inline=\parancite)
block=space,               % kleiner horizontaler Platz zwischen den Feldern
backref=true,              % Seiten anzeigen, auf denen die Referenz vorkommt
backrefstyle=three+,       % fasst Seiten zusammen, z.B. S. 2f, 6ff, 7-10
date=short,                % Datumsformat
giveninits=true,
backend=biber]{biblatex}
\addbibresource{Literatur.bib}
\setlength{\bibitemsep}{10pt}
%Autorenformat ändern
%\DeclareNameFormat{default}{%Vollzitate
%  \ifnum\value{listcount}=1\relax
%    \iffirstinits
%      {\usebibmacro{name:last-first}{#1}{#4}{#5}{#7}}
%      {\usebibmacro{name:last-first}{#1}{#3}{#5}{#7}}%
%  \else
%    \iffirstinits
%      {\usebibmacro{name:first-last}{#1}{#4}{#5}{#7}}
%      {\usebibmacro{name:first-last}{#1}{#3}{#5}{#7}}%
%  \fi
%  \usebibmacro{name:andothers}} 
	

\usepackage[babel,german=quotes]{csquotes}
\AtBeginBibliography{%
  \renewcommand*{\mkbibnamelast}[1]{\textsc{#1}}}
	
\usepackage[colorlinks=true, linkcolor=blue, urlcolor=hell_blau, citecolor=red]{hyperref}


%\renewcommand{\textfraction}{0.0}
%\renewcommand{\topfraction}{1.0}
%\renewcommand{\bottomfraction}{1.0}
%\renewcommand{\thefootnote}{\arabic{footnote})~}
%\renewcommand{\labelenumi}{(\arabic{enumi})}
%%%%%%%%%%%%%%%%%%%%%%%%%%%%%%%%%%%%%%%%%%%%%%%%%%%%%%%%%%%%%%%%%%%%%%%%%%%%%
% 	Section for Page Layout
%%%%%%%%%%%%%%%%%%%%%%%%%%%%%%%%%%%%%%%%%%%%%%%%%%%%%%%%%%%%%%%%%%%%%%%%%%%%%
\singlespacing
%\onehalfspacing
%\doublespacing

\pagestyle{plain}

%%%%%%%%%%%%%%%%%%%%%
% 	all about the text and page style
%		see Latex_layout.svg
%   1. one inch + \hoffset
%   2. one inch + \voffset
%   3. \oddsidemargin = 31pt
%   4. \topmargin = 20pt
%   5. \headheight = 12pt
%   6. \headsep = 25pt
%   7. \textheight = 592pt
%   8. \textwidth = 390pt
%   9. \marginparsep = 10pt
%  10. \marginparwidth = 35pt
%  11. \footskip = 30pt

%		DIN A4 : height x width: 29.7 cm x 21.0 cm
%		1 inch = 2.54 cm
%		define page layout using package "`setlength"'
\setlength{\hoffset}{-0.54cm}  
\setlength{\voffset}{-0.54cm}  
\setlength{\oddsidemargin}{+0.3cm}
\setlength{\evensidemargin}{-0.3cm} 
\setlength{\topmargin}{0.0cm}
\setlength{\headheight}{0.6cm}
\setlength{\headsep}{1cm}
\setlength{\textheight}{23.6cm}
\setlength{\textwidth}{17.0cm}
\setlength{\marginparsep}{0.0cm}  
\setlength{\marginparwidth}{0cm}  
\setlength{\footskip}{1cm}

% additional outlay definitions
%	no indent for new paragraph by \\
\setlength{\parindent}{0mm}
\setlength{\marginparpush}{0cm}  

%%%%%%%%%%%%%%%%%%%%%%%%%%%%%%%%%%%%%%%%%%%%%%%%%%%%%%%%%%%%%%



%%%%%%%%%%%%%%%%%%%%%%%%%%%%%%%%%%%%%%%%%%%%%%%%%%%%%%%%%%%%%%%%%%%%%%%%%%%%%
%    	Set line spacing  (by Multiplier ... 1.5 here)
%%%%%%%%%%%%%%%%%%%%%%%%%%%%%%%%%%%%%%%%%%%%%%%%%%%%%%%%%%%%%%%%%%%%%%%%%%%%%
%\renewcommand{\baselinestretch}{1.05}
%%%%%%%%%%%%%%%%%%%%%%%%%%%%%%%%%%%%%%%%%%%%%%%%%%%%%%%%%%%%%%%%%%%%%%%%%%%%%

%%%%%%%%%%%%%%%%%%%%%%%%%%%%%%%%%%%%%%%%%%%%%%%%%%%%%%%%%%%%%%%%%%%%%%%%%%%%%
%    	Fancy Page Layout with Fancy Header
%%%%%%%%%%%%%%%%%%%%%%%%%%%%%%%%%%%%%%%%%%%%%%%%%%%%%%%%%%%%%%%%%%%%%%%%%%%%%
\pagestyle{fancy}

\let\origdoublepage\cleardoublepage
\newcommand{\clearemptydoublepage}{%
  \clearpage
  {\pagestyle{empty}\origdoublepage}%
}
%%%%%%%%%%%%%%%%%%%%%%%%%%%%%%%%%%%%%%%%%%%%%%%%%%%%%%%%%%%%%%%%%%%%%%%%%%%%%

%%%%%%%%%%%%%%%%%%%%%%%%%%%%%%%%%%%%%%%%%%%%%%%%%%%%%%%%%%%%%%%%%%%%%%%%%%%%%
% 	all about head & foot rule 
\renewcommand{\headrulewidth}{0.5pt}
\renewcommand{\footrulewidth}{0.5pt}
\setlength{\headwidth}{\textwidth}

%\renewcommand{\rightmark}{\thesection\ #1}
%\renewcommand{\leftmark}{Chapter \thechapter}
%\renewcommand{\chaptermark}[1]{\markboth{Chapter~\thechapter:\ #1}{}}
\renewcommand{\sectionmark}[1]{\markboth{\textsc{\thesection.~#1}}{}}
\renewcommand{\subsectionmark}[1]{\markright{\textsc{\thesubsection.~#1}}}
%\renewcommand{\sectionmark}[1]{\markright{\thesection.~#1}{}}
%%%%%%%%%%%%%%%%%%%%%%%%%%%%%%%%%%%%%%%%%%%%%%%%%%%%%%%%%%%%%%%%%%%%%%%%%%%%%

%%%%%%%%%%%%%%%%%%%%%%%%%%%%%%%%%%%%%%%%%%%%%%%%%%%%%%%%%%%%%%%%%%%%%%%%%%%%%
% 	all about shown content in header
%\lhead[\bfseries Page \thepage]{\bfseries\rightmark}
\lhead[\fancyplain{}{\leftmark}]{\fancyplain{}{}}
%\rhead[\bfseries\leftmark]{\bfseries Page \thepage}
\rhead[\fancyplain{}{}]{\fancyplain{}{\rightmark}}
\cfoot{--~\thepage~--}
% \rfoot{\bfseries\thepage}
%%%%%%%%%%%%%%%%%%%%%%%%%%%%%%%%%%%%%%%%%%%%%%%%%%%%%%%%%%%%%%%%%%%%%%%%%%%%%

\fancypagestyle{plain}{%
\fancyhf{} % clear all header and footer fields
\cfoot{--~\thepage~--} % except the center
\renewcommand{\headrulewidth}{0pt}
\renewcommand{\footrulewidth}{0.5pt}}


%%%%%%%%%%%%%%%%%%%%%%%%%%%%%%% AMSTHM new commands %%%%%%%%%%%%%%%%%%%%%%%%%%
%\newtheoremstyle{MyPlain}% name
%  {9pt}%      Space above, empty = `usual value'
% 	{9pt}%      Space below
%  {\itshape}% Body font
%  {\parindent}%         Indent amount (empty = no indent, \parindent = para indent)
%  {\bfseries}% Thm head font
%  {}%        Punctuation after thm head
%  {0.5em}% Space after thm head: \newline = linebreak
%  {}%         Thm head spec 
%
%\theoremstyle{MyPlain} %default[chapter]

%\newtheorem{theorem}{Theorem}[Section]
%\newtheorem{proposition}[theorem]{Proposition}
%\newtheorem{lemma}[theorem]{Lemma}
%\newtheorem{corollary}[theorem]{Corollary}
%\newtheorem{definition}[theorem]{Definition}
%%\newtheorem*{definition*}{Definition}
%\newtheorem{remark}[theorem]{Remark}
%\newtheorem{claim}[theorem]{Claim}
%\newtheorem{example}[theorem]{Example}
%\newtheorem{examples}[theorem]{Examples}
%\newtheorem{assumption}[theorem]{Assumption}
%\newtheorem{motivation}[theorem]{Motivation}
%\newtheorem{property}{Property}
%\newtheorem{note}[theorem]{Note}
 %
%%%%%%%%%%%%%%%%%%%%%%%%%%%%%%% AMSTHM new commands %%%%%%%%%%%%%%%%%%%%%%%%%%
%
%
%%%%%%%%%%%%%%%%%%%%%%%%%%%%%%%%%%%%%%%%%%%%%%%%%%%%%%%%%%%%%%%%%%%%%%%%%%%%%
%	(RE-)DEFINE NEW MATH EXPRESSIONS

%	Define new math operators
%\DeclareMathOperator{\kern}{kern} %somewhere already defined
\DeclareMathOperator{\essinf}{ess-inf}
\DeclareMathOperator*{\esssup}{ess-sup}
\DeclareMathOperator*{\supr}{sup}
\DeclareMathOperator{\dist}{dist}
\DeclareMathOperator{\diag}{diag}
\DeclareMathOperator{\graph}{graph}
\DeclareMathOperator{\sign}{sign}
\DeclareMathOperator{\rank}{rank}
\DeclareMathOperator{\sat}{sat}
\DeclareMathOperator{\spec}{spec}
\DeclareMathOperator{\loc}{loc}
\DeclareMathOperator{\grad}{grad}
\DeclareMathOperator{\fin}{fin}
\DeclareMathOperator{\vol}{vol}
\DeclareMathOperator{\col}{col}
\DeclareMathOperator{\row}{row}
\DeclareMathOperator{\lin}{lin}

%	Abbreviate sets, compact sets  (of numbers, ...)
\newcommand{\N}{\mathbb{N}}
\newcommand{\B}{\mathbb{B}}
\newcommand{\Z}{\mathbb{Z}}
\newcommand{\Q}{\mathbb{Q}}
\newcommand{\R}{\mathbb{R}}
\newcommand{\Rp}{\mathbb{R}_{\geq 0}}
\newcommand{\Rsp}{\mathbb{R}_{> 0}}
\newcommand{\C}{\mathbb{C}}
\newcommand{\Cn}{\mathbb{C}_{< 0}}
\newcommand{\Czp}{\mathbb{C}_{\geq 0}}
%	compact set
\newcommand{\cset}[1]{\mathfrak{#1}}
%	open set
\newcommand{\oset}[1]{\mathcal{#1}}
%	function set
\newcommand{\fset}[1]{\mathcal{#1}}
%	system class 
\newcommand{\sclass}[1]{\mathcal{#1}}
%	operator class 
\newcommand{\oclass}[1]{\mathcal{#1}}
%	boundary set
\newcommand{\bset}[1]{\mathcal{#1}}

%	often used symbols or nomenclature

%	special for Praktikum ET S2
\newcommand{\eff}{\rm{eff}}
\newcommand{\str}{\rm{STR}}


\newcommand{\myref}{\rm{ref}}
%	my (gain) scaling
\newcommand{\gsc}{\varsigma}
\newcommand{\gscmin}{\underline{\varsigma}}
%	high-frequency gain
\newcommand{\hfg}{\gamma_0}
%	steady-state gain
\newcommand{\ssg}{\gamma_{\infty}}
%	root locus center
\newcommand{\rlc}{\Xi}
%	boundaries
%\newcommand{\pF}{\mathcal{F}}
\newcommand{\pF}{\psi}

\newcommand{\pFE}{\psi_E}
\newcommand{\pFphi}{\mathcal{F}_{\phi}}
\newcommand{\dx}[1]{{\rm \, d} #1 \,}
\newcommand{\fdiff}[2]{\frac{\textrm{d}^{#1}}{\textrm{d}#2^{#1}}\,}
\newcommand{\fpartial}[2]{\frac{\partial^{#1}}{\partial #2^{#1}}\,}
\newcommand{\fpartials}[3]{\frac{\partial^{#1}{#3}}{\partial #2^{#1}}\,}
\newcommand{\eps}{\varepsilon}
\newcommand{\foraa}{\textrm{for~a.a.~}}
%	certain instants of time
\newcommand{\tstar}{t^{\star}}
\newcommand{\tlstar}{t_{\star}}
%	motion control objectives
\newcommand{\trise}[1]{t_{y(\cdot),{#1}}^{r}}
\newcommand{\triseref}[1]{t_{{\myref},{#1}}^{r}}
\newcommand{\tset}[1]{t_{y(\cdot),{#1}}^{s}}
\newcommand{\tsetref}[1]{t_{{\myref},{#1}}^{s}}
\newcommand{\os}{\Delta_{y(\cdot)}^{os}}
\newcommand{\osref}{\Delta_{\myref}^{os}}
%	gains 
\newcommand{\kmax}{k_{\max}}
\newcommand{\kinf}{k_{\infty}}
\newcommand{\kmin}{k_{\min}}
\newcommand{\kstar}{k^{\star}}
% control
\newcommand{\ufeas}{u_{\rm{feas}}}
\newcommand{\uhat}{\hat{u}}

\newcommand{\setdef}[2]{\left\{\ #1\ \left|\vphantom{#1}\ #2\ \right.\right\}}

%	function spaces
\newcommand{\Lp}[1]{\fset{L}^{#1}}
\newcommand{\Lone}{\fset{L}^{1}}
\newcommand{\Loneloc}{\fset{L}^{1}_{\loc}}
\newcommand{\Ltwo}{\fset{L}^{2}}
\newcommand{\Linf}{\fset{L}^{\infty}}
\newcommand{\Linfloc}{\fset{L}^{\infty}_{\loc}}
\newcommand{\Ck}{\fset{C}^{k}}
\newcommand{\Czero}{\fset{C}}
\newcommand{\Cone}{\fset{C}^{1}}
\newcommand{\Cinf}{\fset{C}^{\infty}}
\newcommand{\Wkinf}{\fset{W}^{k,\infty}}
\newcommand{\Woneinf}{\fset{W}^{1,\infty}}
\newcommand{\Wtwoinf}{\fset{W}^{2,\infty}}


%	lapace
\newcommand{\laplaceT}[1]{\mathscr{L}\left\{#1\right\}}

%	myVector = \mv
\newcommand{\mv}[1]{\boldsymbol{#1}}
%	myMatrix = \mm
\newcommand{\mm}[1]{\boldsymbol{#1}}
%	myOperator = \mO
\newcommand{\mo}[1]{\boldsymbol{\mathfrak{#1}}}
%	norm ||.||
\newcommand{\norm}[1]{\left\lVert #1 \right\rVert}
% ||.||_infty
\newcommand{\esnorm}[1]{\lVert #1 \rVert_{\infty}}
\newcommand{\Lnorm}[2]{\lVert #2 \rVert_{\fset{L}^{#1}}}
%	inner product <.,.>
\newcommand{\iproduct}[2]{\langle #1,\, #2 \rangle}
%	for dimensions e.g. [Nm]
\newcommand{\di}[1]{\left[\rm{#1}\right]}

% for typewriting:
\newcommand{\tw}[1]{{\tt #1}}

%	system classes
\newcommand{\Sonelin}{\sclass{S}_1^{\lin}}
\newcommand{\Stwolin}{\sclass{S}_2^{\lin}}
\newcommand{\Sone}{\sclass{S}_1}
\newcommand{\Stwo}{\sclass{S}_2}

%	Dimensions
\newcommand{\Ohm}{\Omega}
%
%	CHANGE ENUMERATE NUMBERING
\renewcommand{\labelenumi}{\arabic{enumi}.)}
%\renewcommand{\labelenumii}{(\alph{enumii})}
% 	

% define colors for TIKZ
\definecolor{brown}{RGB}{153, 51,0} %	from Matlab (0.6*255, 0.2*255,0)
\definecolor{orange}{RGB}{255, 127.5,0}
\definecolor{gray}{RGB}{127.5, 127.5,127.5}
%\definecolor{r2}{RGB}{252,177,49}
%\definecolor{r3}{RGB}{35,34,35}
%\definecolor{r4}{RGB}{0,157,87}
%\definecolor{r5}{RGB}{238,50,78}


%	define macros to draw colored lines with TIKZ package
\newcommand{\blueline}{\protect\tikz{\protect\draw[very thick,blue] (0,-0.5ex)(0,0)--(4ex,0);}}
\newcommand{\orangeline}{\protect\tikz{\protect\draw[very thick,orange] (0,-0.5ex)(0,0)--(4ex,0);}}
\newcommand{\brownline}{\protect\tikz{\protect\draw[very thick,brown] (0,-0.5ex)(0,0)--(4ex,0);}}
\newcommand{\bluedashdottedline}{\protect\tikz{\protect\draw[very thick,blue,dashdotted] (0,-0.5ex)(0,0)--(4ex,0);}}
\newcommand{\bluedottedline}{\protect\tikz{\protect\draw[very thick,blue,dotted] (0,-0.5ex)(0,0)--(4ex,0);}}
\newcommand{\bluedashedline}{\protect\tikz{\protect\draw[very thick,dashed,blue] (0,-0.5ex)(0,0)--(4ex,0);}}
\newcommand{\cyanline}{\protect\tikz{\protect\draw[very thick,cyan] (0,-0.5ex)(0,0)--(4ex,0);}}
\newcommand{\cyandashdottedline}{\protect\tikz{\protect\draw[very thick,dashdotted,cyan] (0,-0.5ex)(0,0)--(4ex,0);}}
\newcommand{\cyandashedline}{\protect\tikz{\protect\draw[very thick,dashed,cyan] (0,-0.5ex)(0,0)--(4ex,0);}}
\newcommand{\cyandottedline}{\protect\tikz{\protect\draw[very thick,dotted,cyan] (0,-0.5ex)(0,0)--(4ex,0);}}
\newcommand{\magentaline}{\protect\tikz{\protect\draw[very thick,magenta] (0,-0.5ex)(0,0)--(4ex,0);}}
\newcommand{\magentadashedline}{\protect\tikz{\protect\draw[very thick, dashed,magenta] (0,-0.5ex)(0,0)--(4ex,0);}}
\newcommand{\magentdaottedline}{\protect\tikz{\protect\draw[very thick, dotted,magenta] (0,-0.5ex)(0,0)--(4ex,0);}}
\newcommand{\magentadashdottedline}{\protect\tikz{\protect\draw[very thick, dashdotted,magenta] (0,-0.5ex)(0,0)--(4ex,0);}}
\newcommand{\greenline}{\protect\tikz{\protect\draw[very thick,green] (0,-0.5ex)(0,0)--(4ex,0);}}
\newcommand{\blackline}{\protect\tikz{\protect\draw[very thick,black] (0,-0.5ex)(0,0)--(4ex,0);}}
\newcommand{\blackdashedline}{\protect\tikz{\protect\draw[very thick,dashed,black] (0,-0.5ex)(0,0)--(4ex,0);}}
\newcommand{\blackdottedline}{\protect\tikz{\protect\draw[very thick,dotted,black] (0,-0.5ex)(0,0)--(4ex,0);}}
\newcommand{\blackdashdottedline}{\protect\tikz{\protect\draw[very thick,black,dashdotted] (0,-0.5ex)(0,0)--(4ex,0);}}
\newcommand{\redline}{\protect\tikz{\protect\draw[very thick,red] (0,-0.5ex)(0,0)--(4ex,0);}}
\newcommand{\reddottedline}{\protect\tikz{\protect\draw[very thick,dotted,red] (0,-0.5ex)(0,0)--(4ex,0);}}
\newcommand{\reddashedline}{\protect\tikz{\protect\draw[very thick,dashed,red] (0,-0.5ex)(0,0)--(4ex,0);}}

\newcommand{\graydashedline}{\protect\tikz{\protect\draw[very thick,dashed,gray] (0,-0.5ex)(0,0)--(4ex,0);}}
\newcommand{\graydashdottedline}{\protect\tikz{\protect\draw[very thick,dashdotted,gray] (0,-0.5ex)(0,0)--(4ex,0);}}

%	to include colored comments
\def\ch#1{\textcolor[rgb]{1.00,0.00,0.00}{#1}}

\newcommand{\vect}[1]{\vec{#1}}
\newcommand{\mtx}[1]{{\bf #1}}
\newcommand{\rz}[1]{\underline{#1}} % Raumzeiger
\newcommand{\Clarke}[1]{\mathcal{T_C}\left(#1\right)} % Clarke-Transformation
\newcounter{findex}


\listfiles
\begin{document}
\pagenumbering{roman}

\begin{titlepage}

\vspace*{-2cm}
\begin{minipage}[t]{17.0cm}
	\begin{center}%
		
		\begin{picture}(17.0,3.0)(0,0)
			
			\linethickness{0.04cm}
			
			\put(0.0 ,1.54){\line(1,0){13.3}}
			
			\put(17.2,1.50){\makebox(0,0)[cr]{\includegraphics[height=2.0cm]{./figures/TUM_Logo.eps}}}
			
			\put(0.1,2.73){\makebox(13,0)[tl]{\textbf{Chair of High-Power Converter Systems}}}
			
			\put(0.1,1.71){\makebox(13,0)[bl]{Technical University of Munich \hfill Prof. Marcelo Lobo Heldwein}}
			
			\put(0.1,1.29){\makebox(13,0)[tl]{\renewcommand{\arraystretch}{1.02}
					{\small
						\begin{tabular}[t]{@{}l@{}}%
							Arcisstr. 21            \\
							80333 Munich
					\end{tabular}}
					\hfill
					{\small 
						\begin{tabular}[t]{@{}l@{\,\,\,}l@{}}
							Email: & hlu@ed.tum.de     \\
							Web: & www.epe.ed.tum.de/hlu
					\end{tabular}}
					\hfill
					{\small	
						\begin{tabular}[t]{@{}l@{\,\,\,}l@{}}
							Tel.: & +49 (0)89 289 28358   \\
							Fax:  & +49 (0)89 289 28336
					\end{tabular}}
			}}
			\thinlines
		\end{picture}
	\end{center}
\end{minipage}


\vspace*{3cm}

\begin{center}
{\LARGE \sc Practical Course \\[4ex]}
{\LARGE \bf Simulation and Optimization of  \\[5pt]
Mechatronic Drive Systems for MSPE} \\[8ex]
Summer Semester 2024 
\end{center}

\vfill

\begin{flushleft}

	Haoheng Li / 03797669 \\
\end{flushleft}

\end{titlepage}

\clearpage


%\tableofcontents

\pagenumbering{arabic}
\setcounter{chapter}{1} % command for changing the chapter counter
\chapter{Experiment 2 \\ Control of the DC Drive}

 \setcounter{section}{2} % command for changing the section counter
\section{Armature Current Control}

\begin{enumerate}
\item {\bf *}  Integrator windup occurs when the controller output saturates, causing the integrator keeps accumulating error, which will cause overshoot and instability; it occurs when controller with a integrator such as PI and PID controllers; to prevent it, we can use anti-windup schemes.
\item {\bf *} 
From the differential equations:
\begin{equation}
    \begin{cases}
    U_A = R_A I_A + L_A \frac{dI_A}{dt} + C_M \Psi_{EN} \Omega_M \\
    \Theta_M \frac{d \Omega_M}{dt}=C_M \Psi_{EN} I_A
    \end{cases}
\end{equation}
The transfer function of the DC machine from armature voltage $U_A$ to angular speed $\Omega_M$ is:
\begin{equation}
    F_{DCM}(s) \;=\; \frac{\Omega_M(s)}{U_A(s)} 
= \frac{V_{DCM}}{1 + sT_M + s^2 T_A T_M}
\end{equation}
where $V_{DCM} = \frac{1}{C_M \Psi_{EN}}$, $T_A = \frac{L_A}{R_A}$ and $T_M = \frac{\Theta_M R_A}{(C_M \Psi_{EN})^2}$.

With the numerical values given in the tabel, we have:
\begin{equation}
    F_{DCM}(s) = \frac{1.042}{1 + 0.031 s + 0.000527 s^2}        
\end{equation}

The poles of the system are:
\begin{equation}    
    s_{1,2} = -29.41 \pm j 32.1
\end{equation}

For modulus optimum and symmetrical optimum, $\frac{T_1}{T_\sigma} \geq 4$ and the transfer function above doesn't meet this condition. To make the design by MO and SO, we can introduce a fast inner loop, such as current loop.
\item {\bf *} Introducing EMF $E_A$ acts as a disturbance to the armature current loop. To compensate it, we can use a feedforward control by adding a voltage equal to $E_A$ to the armature voltage reference. The value of compensation voltage should be equal to 
\begin{equation}
  E_{com} =\frac{C_M \Psi_{EN} \Omega_M}{V_{PE}}  
\end{equation}

\end{enumerate}
\subsection{Controller Design}

\begin{enumerate}
\setcounter{enumi}{3} % command for changing the enumeration counter
\item {\bf *} Becasuse the armature current loop is inner loop, it should be designed to be much faster than the speed loop. 
\item {\bf *} 
The transfer function between the armature voltage $U_A$ and the armature current $I_A$ is:
\begin{equation}
    F_{AI}(s) = \frac{I_A(s)}{U_A(s)} = \frac{1/R_A}{1 + sT_A}
\end{equation}

The transfer function of power electronics parts is:
\begin{equation}
    F_{PE}(s) = \frac{U_A(s)}{U^*_A(s)} = \frac{V_{PE}}{1 + sT_{PE}}
\end{equation}

The transfer function of the current sensor is:
\begin{equation}
    F_{IS}(s) = \frac{\hat{I}_A(s)}{I_A(s)} = \frac{1}{1 + sT_{F,I_A}}
\end{equation}

Therefore, the transfer function of the current control loop is:
\begin{equation}
    F_{I_A}(s) = F_{AI}(s) F_{PE}(s) F_{IS}(s)= \frac{\frac{V_{PE}}{R_A}}{(1 + sT_A)(1 + sT_{PE})(1 + sT_{F,I_A})}
\end{equation}

The system above is a third-order system. Becasuse $T_{F,I_A}+T_{PE} \ll T_A$, we can approximate it to a second-order system by neglecting the smallest time constant:
\begin{equation}
    F_{\hat{I}_A}(s) \approx \frac{\frac{V_{PE}}{R_A}}{(1 + sT_A)(1 + sT_{sum,I_A})}
\end{equation}
where $T_{sum,I_A} = T_{F,I_A} + T_{PE}$.

Therefore the approximated transfer function of the current control loop is:
\begin{equation}
    F_{\hat{I}_A}(s)= \frac{V_{S,I_A}}{(1+sT_{1,I_A})(1+sT_{\sigma,I_A})}
\end{equation}
where $V_{S,I_A} = \frac{V_{PE}}{R_A}=0.0455F_{equi,I_A}(s)=\frac{I_A(s)}{I_A^*(s)}\approx\frac{1}{1+sT_{equi,I_A}}$, $T_{1,I_A} = T_A$ and $T_{\sigma,I_A} = T_{sum,I_A}=2.5ms$.
\item {\bf *} From the optimization table, we can design the PI controller by modulus optimum criterion:
\begin{equation}
    F_{C,I_A}(s)=V_{\mathrm{r},I_A}\frac{1+sT_{\mathrm{n},I_A}}{sT_{\mathrm{n},I_A}}
\end{equation}
where $T_{\mathrm{n},I_A} = T_{\sigma,I_A}=T_A=17ms$, and $V_{\mathrm{r},I_A} = \frac{T_{1,I_A}}{2V_{S,I_A}T_{\sigma,I_A}} = 74.8$. 
\end{enumerate}

\section{Speed Control in the Armature Control Range}
\subsection{Controller Design}
\begin{enumerate}
\setcounter{enumi}{10} % command for changing the enumeration counter
\item {\bf *} 
The closed-loop transfer function of the curren control loop is:
\begin{equation}
    F_{equi,I_A}(s) = \frac{\hat{I}_A(s)}{I_A^*(s)} = \frac{F_{C,I_A}(s)F_{PE}(s) F_{AI}(s) }{1 + F_{C,I_A}(s)F_{PE}(s) F_{AI}(s) F_{IS}(s)}=\frac{1+sT_{f,I_{A}}}{1+sT_{\sigma,I_{A}}+s^{2}2T_{\sigma,I_{A}}^{2}}
\end{equation}
\item {\bf *} If we  apply a polynomial division by the numerator, we can get:
\begin{equation}
    F_{equi,I_A}(s)=\frac{I_A(s)}{I_A^*(s)}\approx\frac{1}{1+sT_{equi,I_A}}
\end{equation}
where $T_{equi,I_A} = 2T_{\sigma,I_A} - T_{f,I_A} = 3ms$.
\item {\bf *}   
From the differential equations, we can derive the transfer function from the armature current reference $I^*_A$ to the filtered angular speed $\hat{\Omega}_M$:
\begin{equation}
    F_{\hat{\Omega}_M}(s) = \frac{\hat{\Omega}_M(s)}{I_A^*(s)} = F_{\hat{\Omega}_M,I_A}(s) \cdot \frac{1}{\Theta_M s} \cdot C_M \Psi_{EN} \cdot F_{equi,I_A}(s)
\end{equation}

Simplify it to the standard form, we have:
\begin{equation}
    F_{\hat{\Omega}_M}(s)=\frac{\hat{\Omega}_M(s)}{I_A^*(s)}=\frac{V_{S,\Omega_M}}{sT_{1,\Omega_M}(1+sT_{\sigma,\Omega_M})}
\end{equation}
where $V_{S,\Omega_M} = \frac{R_A}{C_M \Psi_{EN}}=22.92$, $T_{1,\Omega_M} = T_M=\frac{R_A\Theta_M}{C_M^2\Psi_{EN}^2}=31ms$ and $T_{\sigma,\Omega_M} = T_{equi,I_A}+T_{F,\Omega_M}=5ms$.
\item {\bf *}
The above system is a $IT_1$ system. From the optimization table, we can design the PI controller by symmetrical optimum criterion:
\begin{equation}
    F_{C,\Omega_M}(s)=V_{\mathrm{r},\Omega_M}\frac{1+sT_{\mathrm{n},\Omega_M}}{sT_{\mathrm{n},\Omega_M}}
\end{equation}
where $T_{\mathrm{n},\Omega_M} = 4T_{\sigma,\Omega_M}=20ms$, and $V_{\mathrm{r},\Omega_M} = \frac{T_{1,\Omega_M}}{2V_{S,\Omega_M}T_{\sigma,\Omega_M}}=0.135$.
\end{enumerate}


\end{document}
