%%%%%%%%%%%%%%%%%%%%%%%%%%%%%%% AMSTHM new commands %%%%%%%%%%%%%%%%%%%%%%%%%%
%\newtheoremstyle{MyPlain}% name
%  {9pt}%      Space above, empty = `usual value'
% 	{9pt}%      Space below
%  {\itshape}% Body font
%  {\parindent}%         Indent amount (empty = no indent, \parindent = para indent)
%  {\bfseries}% Thm head font
%  {}%        Punctuation after thm head
%  {0.5em}% Space after thm head: \newline = linebreak
%  {}%         Thm head spec 
%
%\theoremstyle{MyPlain} %default[chapter]

%\newtheorem{theorem}{Theorem}[Section]
%\newtheorem{proposition}[theorem]{Proposition}
%\newtheorem{lemma}[theorem]{Lemma}
%\newtheorem{corollary}[theorem]{Corollary}
%\newtheorem{definition}[theorem]{Definition}
%%\newtheorem*{definition*}{Definition}
%\newtheorem{remark}[theorem]{Remark}
%\newtheorem{claim}[theorem]{Claim}
%\newtheorem{example}[theorem]{Example}
%\newtheorem{examples}[theorem]{Examples}
%\newtheorem{assumption}[theorem]{Assumption}
%\newtheorem{motivation}[theorem]{Motivation}
%\newtheorem{property}{Property}
%\newtheorem{note}[theorem]{Note}
 %
%%%%%%%%%%%%%%%%%%%%%%%%%%%%%%% AMSTHM new commands %%%%%%%%%%%%%%%%%%%%%%%%%%
%
%
%%%%%%%%%%%%%%%%%%%%%%%%%%%%%%%%%%%%%%%%%%%%%%%%%%%%%%%%%%%%%%%%%%%%%%%%%%%%%
%	(RE-)DEFINE NEW MATH EXPRESSIONS

%	Define new math operators
%\DeclareMathOperator{\kern}{kern} %somewhere already defined
\DeclareMathOperator{\essinf}{ess-inf}
\DeclareMathOperator*{\esssup}{ess-sup}
\DeclareMathOperator*{\supr}{sup}
\DeclareMathOperator{\dist}{dist}
\DeclareMathOperator{\diag}{diag}
\DeclareMathOperator{\graph}{graph}
\DeclareMathOperator{\sign}{sign}
\DeclareMathOperator{\rank}{rank}
\DeclareMathOperator{\sat}{sat}
\DeclareMathOperator{\spec}{spec}
\DeclareMathOperator{\loc}{loc}
\DeclareMathOperator{\grad}{grad}
\DeclareMathOperator{\fin}{fin}
\DeclareMathOperator{\vol}{vol}
\DeclareMathOperator{\col}{col}
\DeclareMathOperator{\row}{row}
\DeclareMathOperator{\lin}{lin}

%	Abbreviate sets, compact sets  (of numbers, ...)
\newcommand{\N}{\mathbb{N}}
\newcommand{\B}{\mathbb{B}}
\newcommand{\Z}{\mathbb{Z}}
\newcommand{\Q}{\mathbb{Q}}
\newcommand{\R}{\mathbb{R}}
\newcommand{\Rp}{\mathbb{R}_{\geq 0}}
\newcommand{\Rsp}{\mathbb{R}_{> 0}}
\newcommand{\C}{\mathbb{C}}
\newcommand{\Cn}{\mathbb{C}_{< 0}}
\newcommand{\Czp}{\mathbb{C}_{\geq 0}}
%	compact set
\newcommand{\cset}[1]{\mathfrak{#1}}
%	open set
\newcommand{\oset}[1]{\mathcal{#1}}
%	function set
\newcommand{\fset}[1]{\mathcal{#1}}
%	system class 
\newcommand{\sclass}[1]{\mathcal{#1}}
%	operator class 
\newcommand{\oclass}[1]{\mathcal{#1}}
%	boundary set
\newcommand{\bset}[1]{\mathcal{#1}}

%	often used symbols or nomenclature

%	special for Praktikum ET S2
\newcommand{\eff}{\rm{eff}}
\newcommand{\str}{\rm{STR}}


\newcommand{\myref}{\rm{ref}}
%	my (gain) scaling
\newcommand{\gsc}{\varsigma}
\newcommand{\gscmin}{\underline{\varsigma}}
%	high-frequency gain
\newcommand{\hfg}{\gamma_0}
%	steady-state gain
\newcommand{\ssg}{\gamma_{\infty}}
%	root locus center
\newcommand{\rlc}{\Xi}
%	boundaries
%\newcommand{\pF}{\mathcal{F}}
\newcommand{\pF}{\psi}

\newcommand{\pFE}{\psi_E}
\newcommand{\pFphi}{\mathcal{F}_{\phi}}
\newcommand{\dx}[1]{{\rm \, d} #1 \,}
\newcommand{\fdiff}[2]{\frac{\textrm{d}^{#1}}{\textrm{d}#2^{#1}}\,}
\newcommand{\fpartial}[2]{\frac{\partial^{#1}}{\partial #2^{#1}}\,}
\newcommand{\fpartials}[3]{\frac{\partial^{#1}{#3}}{\partial #2^{#1}}\,}
\newcommand{\eps}{\varepsilon}
\newcommand{\foraa}{\textrm{for~a.a.~}}
%	certain instants of time
\newcommand{\tstar}{t^{\star}}
\newcommand{\tlstar}{t_{\star}}
%	motion control objectives
\newcommand{\trise}[1]{t_{y(\cdot),{#1}}^{r}}
\newcommand{\triseref}[1]{t_{{\myref},{#1}}^{r}}
\newcommand{\tset}[1]{t_{y(\cdot),{#1}}^{s}}
\newcommand{\tsetref}[1]{t_{{\myref},{#1}}^{s}}
\newcommand{\os}{\Delta_{y(\cdot)}^{os}}
\newcommand{\osref}{\Delta_{\myref}^{os}}
%	gains 
\newcommand{\kmax}{k_{\max}}
\newcommand{\kinf}{k_{\infty}}
\newcommand{\kmin}{k_{\min}}
\newcommand{\kstar}{k^{\star}}
% control
\newcommand{\ufeas}{u_{\rm{feas}}}
\newcommand{\uhat}{\hat{u}}

\newcommand{\setdef}[2]{\left\{\ #1\ \left|\vphantom{#1}\ #2\ \right.\right\}}

%	function spaces
\newcommand{\Lp}[1]{\fset{L}^{#1}}
\newcommand{\Lone}{\fset{L}^{1}}
\newcommand{\Loneloc}{\fset{L}^{1}_{\loc}}
\newcommand{\Ltwo}{\fset{L}^{2}}
\newcommand{\Linf}{\fset{L}^{\infty}}
\newcommand{\Linfloc}{\fset{L}^{\infty}_{\loc}}
\newcommand{\Ck}{\fset{C}^{k}}
\newcommand{\Czero}{\fset{C}}
\newcommand{\Cone}{\fset{C}^{1}}
\newcommand{\Cinf}{\fset{C}^{\infty}}
\newcommand{\Wkinf}{\fset{W}^{k,\infty}}
\newcommand{\Woneinf}{\fset{W}^{1,\infty}}
\newcommand{\Wtwoinf}{\fset{W}^{2,\infty}}


%	lapace
\newcommand{\laplaceT}[1]{\mathscr{L}\left\{#1\right\}}

%	myVector = \mv
\newcommand{\mv}[1]{\boldsymbol{#1}}
%	myMatrix = \mm
\newcommand{\mm}[1]{\boldsymbol{#1}}
%	myOperator = \mO
\newcommand{\mo}[1]{\boldsymbol{\mathfrak{#1}}}
%	norm ||.||
\newcommand{\norm}[1]{\left\lVert #1 \right\rVert}
% ||.||_infty
\newcommand{\esnorm}[1]{\lVert #1 \rVert_{\infty}}
\newcommand{\Lnorm}[2]{\lVert #2 \rVert_{\fset{L}^{#1}}}
%	inner product <.,.>
\newcommand{\iproduct}[2]{\langle #1,\, #2 \rangle}
%	for dimensions e.g. [Nm]
\newcommand{\di}[1]{\left[\rm{#1}\right]}

% for typewriting:
\newcommand{\tw}[1]{{\tt #1}}

%	system classes
\newcommand{\Sonelin}{\sclass{S}_1^{\lin}}
\newcommand{\Stwolin}{\sclass{S}_2^{\lin}}
\newcommand{\Sone}{\sclass{S}_1}
\newcommand{\Stwo}{\sclass{S}_2}

%	Dimensions
\newcommand{\Ohm}{\Omega}
%
%	CHANGE ENUMERATE NUMBERING
\renewcommand{\labelenumi}{\arabic{enumi}.)}
%\renewcommand{\labelenumii}{(\alph{enumii})}
% 	

% define colors for TIKZ
\definecolor{brown}{RGB}{153, 51,0} %	from Matlab (0.6*255, 0.2*255,0)
\definecolor{orange}{RGB}{255, 127.5,0}
\definecolor{gray}{RGB}{127.5, 127.5,127.5}
%\definecolor{r2}{RGB}{252,177,49}
%\definecolor{r3}{RGB}{35,34,35}
%\definecolor{r4}{RGB}{0,157,87}
%\definecolor{r5}{RGB}{238,50,78}


%	define macros to draw colored lines with TIKZ package
\newcommand{\blueline}{\protect\tikz{\protect\draw[very thick,blue] (0,-0.5ex)(0,0)--(4ex,0);}}
\newcommand{\orangeline}{\protect\tikz{\protect\draw[very thick,orange] (0,-0.5ex)(0,0)--(4ex,0);}}
\newcommand{\brownline}{\protect\tikz{\protect\draw[very thick,brown] (0,-0.5ex)(0,0)--(4ex,0);}}
\newcommand{\bluedashdottedline}{\protect\tikz{\protect\draw[very thick,blue,dashdotted] (0,-0.5ex)(0,0)--(4ex,0);}}
\newcommand{\bluedottedline}{\protect\tikz{\protect\draw[very thick,blue,dotted] (0,-0.5ex)(0,0)--(4ex,0);}}
\newcommand{\bluedashedline}{\protect\tikz{\protect\draw[very thick,dashed,blue] (0,-0.5ex)(0,0)--(4ex,0);}}
\newcommand{\cyanline}{\protect\tikz{\protect\draw[very thick,cyan] (0,-0.5ex)(0,0)--(4ex,0);}}
\newcommand{\cyandashdottedline}{\protect\tikz{\protect\draw[very thick,dashdotted,cyan] (0,-0.5ex)(0,0)--(4ex,0);}}
\newcommand{\cyandashedline}{\protect\tikz{\protect\draw[very thick,dashed,cyan] (0,-0.5ex)(0,0)--(4ex,0);}}
\newcommand{\cyandottedline}{\protect\tikz{\protect\draw[very thick,dotted,cyan] (0,-0.5ex)(0,0)--(4ex,0);}}
\newcommand{\magentaline}{\protect\tikz{\protect\draw[very thick,magenta] (0,-0.5ex)(0,0)--(4ex,0);}}
\newcommand{\magentadashedline}{\protect\tikz{\protect\draw[very thick, dashed,magenta] (0,-0.5ex)(0,0)--(4ex,0);}}
\newcommand{\magentdaottedline}{\protect\tikz{\protect\draw[very thick, dotted,magenta] (0,-0.5ex)(0,0)--(4ex,0);}}
\newcommand{\magentadashdottedline}{\protect\tikz{\protect\draw[very thick, dashdotted,magenta] (0,-0.5ex)(0,0)--(4ex,0);}}
\newcommand{\greenline}{\protect\tikz{\protect\draw[very thick,green] (0,-0.5ex)(0,0)--(4ex,0);}}
\newcommand{\blackline}{\protect\tikz{\protect\draw[very thick,black] (0,-0.5ex)(0,0)--(4ex,0);}}
\newcommand{\blackdashedline}{\protect\tikz{\protect\draw[very thick,dashed,black] (0,-0.5ex)(0,0)--(4ex,0);}}
\newcommand{\blackdottedline}{\protect\tikz{\protect\draw[very thick,dotted,black] (0,-0.5ex)(0,0)--(4ex,0);}}
\newcommand{\blackdashdottedline}{\protect\tikz{\protect\draw[very thick,black,dashdotted] (0,-0.5ex)(0,0)--(4ex,0);}}
\newcommand{\redline}{\protect\tikz{\protect\draw[very thick,red] (0,-0.5ex)(0,0)--(4ex,0);}}
\newcommand{\reddottedline}{\protect\tikz{\protect\draw[very thick,dotted,red] (0,-0.5ex)(0,0)--(4ex,0);}}
\newcommand{\reddashedline}{\protect\tikz{\protect\draw[very thick,dashed,red] (0,-0.5ex)(0,0)--(4ex,0);}}

\newcommand{\graydashedline}{\protect\tikz{\protect\draw[very thick,dashed,gray] (0,-0.5ex)(0,0)--(4ex,0);}}
\newcommand{\graydashdottedline}{\protect\tikz{\protect\draw[very thick,dashdotted,gray] (0,-0.5ex)(0,0)--(4ex,0);}}

%	to include colored comments
\def\ch#1{\textcolor[rgb]{1.00,0.00,0.00}{#1}}

\newcommand{\vect}[1]{\vec{#1}}
\newcommand{\mtx}[1]{{\bf #1}}
\newcommand{\rz}[1]{\underline{#1}} % Raumzeiger
\newcommand{\Clarke}[1]{\mathcal{T_C}\left(#1\right)} % Clarke-Transformation
\newcounter{findex}

